\documentclass{ctexart}

\usepackage{geometry}
\usepackage{enumitem}
\usepackage{minted}
\usepackage{mdframed}
\usepackage{graphicx}

\setitemize{noitemsep,topsep=0pt,parsep=0pt,partopsep=0pt}
\setenumerate{noitemsep,topsep=0pt,parsep=0pt,partopsep=0pt}

\title{\heiti 《数据库系统概论》课程项目实验报告}
\author{\kaishu 邢竞择 2020012890}

\geometry{top=3cm,bottom=3cm}

\begin{document}
\maketitle
\section{整体设计}
\subsection{基本设计}
项目使用 C++ 编写,依赖于 antlr4-runtime 和 antlr4-cpp 这两个外部库进行语法解析,使用 argparse 解析命令行参数,此外还使用 googletest 编写了 B+ 树的功能测试。
\subsection{模块设计}
模块设计基本上参照实验指导书,分为文件管理模块、记录管理模块、索引管理模块、SQL 解析模块和查询处理模块。文件管理模块的,SQL 解析模块的实现在 \texttt{src/frontend} 目录下,其余三个模块的实现均在 \texttt{src/engine} 目录下。
\subsection{功能实现}
除了全部基本功能外,本项目还实现了具有 CI 测例的全部附加功能:
\begin{itemize}
	\item 多表 join
	\item 聚合查询、模糊查询、分组查询、嵌套查询、查询结果排序
	\item 日期、完整 null 支持
	\item unique 约束
\end{itemize}

\section{具体设计}
\subsection{工具模块}
该模块的代码实现在\texttt{src/utils}目录下。为了方便参数管理,我实现了一个单例模式的\texttt{Config}类用于管理全部配置,它负责解析命令行参数并提供给其他模块使用。

为了方便统一数据库的输出,我实现了一系列\texttt{tabulate}函数,能够按照当前交互模式(交互或者批处理)来输出查询结果。

此外,这部分还提供诸如\texttt{ensure\_file}和\texttt{ensure\_directory}的函数,方便在其他模块中检查文件和目录,提升代码健壮性。

\subsection{文件管理模块}
该模块的代码实现在 \texttt{src/storage} 目录下。这部分代码在功能上与提供的参考代码类似,但是我使用更加现代化的 C++ 实现进行了优化。

具体而言,文件管理系统实现为单例模式的\texttt{FileMapping}类,除了正常文件的打开关闭以外,还支持创建和删除临时文件(利用\texttt{mkstemp})。在销毁时,它会关闭所有文件并自动删除临时文件。

分页缓存池被实现成单例模式的\texttt{PagedBuffer}类,它通过\texttt{FileMapping}管理文件,支持按需读取和写回,支持 LRU 策略的缓存替换。在被销毁时它会自动将全部脏页写入文件中,它通过\texttt{std::shared\_ptr}来确保\texttt{FileMapping}在它之后销毁。

\texttt{SequentialAccessor}用于实现顺序的读写,它内部保存一个\texttt{fd}和一个\texttt{offset},能显著地方便上层实现。例如,当需要将表的元数据写入文件时,表可以将\texttt{SequentialAccessor}的引用传给属于它的\texttt{Fields},\texttt{Fields}就可以直接将自己的数据写入文件。

此外,由于 B+ 树本身与数据库模块解耦,因此将其放在文件管理模块中,它提供各种查询接口供索引管理模块使用。为了确保 B+ 树实现正确,我实现了一个 googletest 测例,它对 B+ 树进行插入、删除、查找等操作,并检查其正确性,详细实现位于\texttt{test
	/test\_btree.cpp}。

\subsection{记录管理模块}
该模块用于实现记录存储,代码位于\texttt{src/engine/record.cpp}中。记录数据分页存储,每一页大小为 8 KB,每页开头记录该页有效记录的数量、下一个空页的页号,紧跟着是一个\texttt{FixedBitmap},用于记录每一个 slot 是否被使用,数据采用定长类型进行存储。记录长度、每页最大记录数量、第一个空页的页号等数据与表的元数据存储在一起。
\texttt{RecordManager}实现了对单张表记录读写的管理。提供了数据插入、删除、查找、修改的接口。

\subsection{索引管理模块}
该模块用于实现索引存储,代码位于\texttt{src/engine/index.cpp}中。\texttt{IndexMeta}包含一个 B+ 树和一个引用计数,当引用计数为 0 时,B+ 树会被销毁。数据库进程结束时,主键、外键、unique 约束等在存储时仅需标记它们取了哪些列,下次启动时,数据库仅需按照这些列找到对应的\texttt{IndexMeta}即可。

\subsection{查询处理模块}
查询的执行基于迭代器实现;查询中的限制检查和修改操作分别由\texttt{WhereConstraint}和\texttt{SetVariable}执行。对于增删改操作,需要对各种约束情况加以考虑。以下详细介绍各个部分的实现思路。

\subsubsection{迭代器}
迭代器从基本功能上分为两类,一类从文件(记录和索引)中读取数据(并进行筛选),另一类在前一类的基础上进行连接、过滤、聚合等操作。而从具体优化上,可以允许迭代器一次从文件中读取多条记录(可以称为“带缓冲的分块读取”),这样在连接时可以采用分块的优化,减少磁盘 IO 次数。在实现时,迭代器往往需要记录它迭代的来源的列信息、它提供的结果的列信息、当前迭代的位置、传入的约束等信息。本项目实现了以下几种迭代器。
\begin{itemize}
	\item \texttt{RecordIterator}访问记录文件,分块读取
	\item \texttt{IndexIterator}访问索引文件,分块读取
	\item \texttt{JoinIterator}连接两个迭代器,这两个迭代器需要支持分块读取
	\item \texttt{PermuteIterator}对上一个迭代器的列进行重排
	\item \texttt{AggregateIterator}对上一个迭代器的结果进行聚合操作,在执行聚合时会顺便进行列重排
	\item \texttt{SortIterator}对上一个迭代器的结果按照某一关键词进行排序
\end{itemize}
迭代器之间构成了一个树的结构,数据先由\texttt{RecordIterator}或者\texttt{IndexIterator}进行提取,然后由\texttt{JoinIterator}连接,接着通过\texttt{PermuteIterator}或\texttt{AggregateIterator}进行列重排(或进行聚合),最后如需排序,则通过\texttt{SortIterator}。输出函数不断地从最顶部的迭代器取出元素,使各个迭代器不断地产生、筛选新的记录,整个过程类似于流水线。

\texttt{QueryPlanner}的功能是按照查询语句构建迭代器树,当\texttt{GROUP BY}所需的列不在查询的列中时,它负责临时将其加入;\texttt{LIMIT}和\texttt{OFFSET}限制也由它进行处理。如果某一个查询是一个子查询,那么它上层查询的\texttt{WhereConstraint}会基于它的\texttt{QueryPlanner}执行;否则,\texttt{QueryPlanner}会直接送给输出模块。

\subsubsection{查询描述、执行}
\texttt{WhereConstraint}是一个基类,它要求所有子类实现一个\texttt{check}函数来检查某条记录是否满足要求,它的子类包括
\begin{itemize}
	\item \texttt{ColumnOpValueConstraint}表示列和值之间的比较关系
	\item \texttt{ColumnOpColumnConstraint}表示列之间的比较关系
	\item \texttt{ColumnNullConstraint}进行\texttt{NULL}检查
	\item \texttt{ColumnLikeStringConstraint}进行字符串模糊查询
	\item \texttt{ColumnOpSubqueryConstraint}列和子查询的比较关系
	\item \texttt{ColumnInSubqueryConstraint}列和子查询的包含关系
\end{itemize}

\texttt{SetVariable}类中记录了待修改的列的offset,以及需要修改成的值,它能且仅能对一条记录进行修改,关于约束的检查由上层代码完成。

\subsubsection{增删改操作}
为了方便调用,增删改操作在\texttt{src/engine/scape\_sql.cpp}中实现,这样前端在解析 SQL 语句时可以直接调用这些封装好的函数。另外,为了优化外键约束的检查,我在 B+ 树的实现中,允许叶节点存储一个\texttt{int32\_t}类型的引用计数来表示这条记录被外键引用的次数。

对于\texttt{INSERT}操作,检查\texttt{PRIMARY,FOREIGN,UNIQUE}约束是否满足,若满足,则向记录以及每个\texttt{IndexMeta}管理的 B+ 树插入数据,并更新它引用的外键的引用计数。

对于\texttt{DELETE}操作,检查\texttt{FOREIGN}约束是否满足(即引用计数是否为 0),若满足要求,则可在索引和记录中删去。

对于\texttt{UPDATE}操作,虽然它等价于一次删除和一次插入,但主外键约束使得情况复杂一些。如果修改前后主键对应的列数值不变,那么只要检查外键约束是否满足即可(这就允许它的引用计数大于零);否则,则可以等价为一次删除和一次插入,如果无法删除或无法插入,则恢复原状并报错。

\subsection{系统管理模块}
系统管理模块包括\texttt{GlobalManager,DatabaseManager,TableManager}三个层级,代码在\texttt{src/engine/system.cpp}中。它们都具有元数据的存取功能,而对于\texttt{TableManager},它还提供索引、主外键约束、unique 约束的增删功能。

\subsection{SQL 语句解析模块}
该模块使用 antlr4 进行实现,代码位于\texttt{src/frontend}目录下。它首先使用 antlr4 生成语法分析树,接着使用自定义的\texttt{Visitor}访问语法树,并在访问过程中实时地执行操作。

\section{接口设计}
% use minted to highlight code, use \begin{minted}{cpp} to start
\subsection{文件管理模块}
\begin{mdframed}
	\begin{minted}[fontsize=\small]{cpp}
class PagedBuffer {
  // 为实现 LRU 刷新策略,使用双向链表
  void access(int id);
  // 根据指定的文件描述符和页号,从文件中读取一页
  uint8_t *read_file_rd(PageLocator pos);
  // 读取后将该页标记为脏页
  uint8_t *read_file_rdwr(PageLocator pos);
  // 将指定页标记为脏页,由于页的分配是连续的,可由指针判断属于哪个页
  bool mark_dirty(uint8_t *ptr);
};
class SequentialAccessor {
  void reset(int pagenum_);
  // 读取接口
  uint8_t read_byte();
  template <typename T> T read();
  std::string read_str();
  // 写入接口
  void write_byte(uint8_t byte);
  template <typename T> void write(T val);
  void write_str(const std::string &val);
};
\end{minted}
\end{mdframed}

\subsection{记录管理模块}

\begin{mdframed}
	\begin{minted}[fontsize=\small]{cpp}
class RecordManager {
  // 创建表时使用的构造函数
  RecordManager(const std::string &datafile_name, int record_len);
  // 从文件中读取元数据的构造函数
  RecordManager(SequentialAccessor &accessor);
  // 元数据文件写入
  void serialize(SequentialAccessor &accessor);
  // 其他一些模块也有这样一组构造函数和析构函数,为了简洁,未来不再赘述

  // 获取指定位置的记录
  uint8_t *get_record_ref(int pageid, int slotid);
  // 插入记录并返回它的位置(页号和槽号)给索引使用
  std::pair<int, int> insert_record(const uint8_t *ptr);
  // 删除记录
  void erase_record(int pagenum, int slotnum);
};
\end{minted}
\end{mdframed}

\subsection{索引管理模块}
\begin{mdframed}
	\begin{minted}[fontsize=\small]{cpp}
struct IndexMeta {
  // IndexMeta 本质上是 B+ 树的封装,便于从记录中提取键值
  // 为一个表建立外键约束时,它的外键 Offset 与被引用表的主键 Offset 不同
  // 因此需要通过 remap 函数重新建立偏移量
  std::shared_ptr<IndexMeta>
    remap(const std::vector<std::shared_ptr<Field>> &keys) const;
  // 从记录中提取键值
  std::vector<int> extractKeys(const KeyCollection &data);
  // 插入记录
  void insert_record(KeyCollection data);
  // 小于等于匹配,上层可以利用它查询是否出现过某个 key
  BPlusQueryResult le_match(KeyCollection data);
  // 获取引用计数
  uint32_t *get_refcount(uint8_t *ptr);
};
\end{minted}
\end{mdframed}

\subsection{迭代器}
\begin{mdframed}
	\begin{minted}[fontsize=\small]{cpp}
class Iterator {
  virtual bool get_next_valid() = 0;
  virtual const uint8_t *get() const = 0;
};
class BlockIterator : public Iterator {
  // 对迭代器缓存的结果块进行操作、查询
  void block_next();
  bool block_end();
  bool all_end();
  // 让迭代器筛选出下一块记录,返回填入的记录数量
  virtual int fill_next_block() = 0;
  // 查询当前块/全部记录是否访问结束
  virtual void reset_all() = 0;
  void reset_block();
};
class RecordIterator : public BlockIterator;
class IndexIterator : public BlockIterator;
class JoinIterator : public BlockIterator;
class PermuteIterator : public Iterator;
// Gather 迭代器将打断流水线,它只有在将上层迭代器的结果全部读取后才能得到新的结果
class GatherIterator : public Iterator;
class AggregateIterator : public GatherIterator;
class SortIterator : public GatherIterator;

class QueryPlanner {
  // QueryPlanner 有大量的成员变量,调用前需要逐个填入,最后调用 generate_plan 生成迭代器
  void generate_plan();
  // 获取一条记录
  const uint8_t *get() const;
  // 进入下一条记录
  bool next();
};
\end{minted}
\end{mdframed}

\subsection{SQL 执行接口}
前文提到,数据库语句执行被包装成方便调用的一系列接口,它们能使代码更加整洁,并且可以方便地确保代码的健壮性,具体定义如下:
\begin{mdframed}
	\begin{minted}[fontsize=\small]{cpp}
namespace ScapeSQL {
// 简单数据库管理操作
void create_db(const std::string &s);
void drop_db(const std::string &s);
void show_dbs();
void use_db(const std::string &s);
void show_tables();
void show_indexes();
void create_table(const std::string &s,
                  std::vector<std::shared_ptr<Field>> &&fields);
void drop_table(const std::string &s);
void describe_table(const std::string &s);
// 增删改查操作
void update_set_table(
    std::shared_ptr<TableManager> table,
    std::vector<SetVariable> &&set_variables,
    std::vector<std::shared_ptr<WhereConstraint>> &&where_constraints);
void delete_from_table(
    std::shared_ptr<TableManager> table,
    std::vector<std::shared_ptr<WhereConstraint>> &&where_constraints);
void insert_from_file(const std::string &file_path, const std::string &table_name);
// 增加、删除约束
void add_pk(const std::string &table_name, std::shared_ptr<PrimaryKey> key);
void drop_pk(const std::string &table_name, const std::string &pk_name);
void add_fk(const std::string &table_name, std::shared_ptr<ForeignKey> key);
void drop_fk(const std::string &table_name, const std::string &fk_name);
void add_index(const std::string &table_name, std::shared_ptr<ExplicitIndexKey> key);
void drop_index(const std::string &table_name, const std::string &index_name);
void add_unique(const std::string &table_name, std::shared_ptr<UniqueKey> key);
} // namespace ScapeSQL
\end{minted}
\end{mdframed}

\section{实验结果}
本课程项目能够通过全部 CI 测例,在本机(CPU: R7 7840HS)下,限制分页缓存为\texttt{64MB},能够在 10 秒内完成全部测试。以下是本机测试截图:
\begin{figure}[htbp]
  \centering
  \includegraphics[width=1.0\textwidth]{result.png}
\end{figure}

本人能力有限,在最初设计时没有考虑到变长字符串、增删列的需求,导致难以在现有代码框架上实现这些功能,故没有完成这一部分扩展内容。

\section{参考文献}
\begin{enumerate}
  \item 课程组提供的实验指导书
  \item Antlr4 教程(\texttt{https://tomassetti.me/antlr-mega-tutorial/})
  \item C++ 正则表达式教程(\texttt{https://en.cppreference.com/w/cpp/regex})
\end{enumerate}

\end{document}